\documentclass[journal,12pt,twocolumn]{IEEEtran}

\usepackage{setspace}
\usepackage{gensymb}

\singlespacing


\usepackage[cmex10]{amsmath}

\usepackage{amsthm}

\usepackage{mathrsfs}
\usepackage{txfonts}
\usepackage{stfloats}
\usepackage{bm}
\usepackage{cite}
\usepackage{cases}
\usepackage{subfig}

\usepackage{longtable}
\usepackage{multirow}

\usepackage{enumitem}
\usepackage{mathtools}
\usepackage{steinmetz}
\usepackage{tikz}
\usepackage{circuitikz}
\usepackage{verbatim}
\usepackage{tfrupee}
\usepackage[breaklinks=true]{hyperref}
\usepackage{graphicx}
\usepackage{tkz-euclide}

\usetikzlibrary{calc,math}
\usepackage{listings}
    \usepackage{color}                                            %%
    \usepackage{array}                                            %%
    \usepackage{longtable}                                        %%
    \usepackage{calc}                                             %%
    \usepackage{multirow}                                         %%
    \usepackage{hhline}                                           %%
    \usepackage{ifthen}                                           %%
    \usepackage{lscape}     
\usepackage{multicol}
\usepackage{chngcntr}

\DeclareMathOperator*{\Res}{Res}

\renewcommand\thesection{\arabic{section}}
\renewcommand\thesubsection{\thesection.\arabic{subsection}}
\renewcommand\thesubsubsection{\thesubsection.\arabic{subsubsection}}

\renewcommand\thesectiondis{\arabic{section}}
\renewcommand\thesubsectiondis{\thesectiondis.\arabic{subsection}}
\renewcommand\thesubsubsectiondis{\thesubsectiondis.\arabic{subsubsection}}


\hyphenation{op-tical net-works semi-conduc-tor}
\def\inputGnumericTable{}                                 %%

\lstset{
%language=C,
frame=single, 
breaklines=true,
columns=fullflexible
}
\begin{document}


\newtheorem{theorem}{Theorem}[section]
\newtheorem{problem}{Problem}
\newtheorem{proposition}{Proposition}[section]
\newtheorem{lemma}{Lemma}[section]
\newtheorem{corollary}[theorem]{Corollary}
\newtheorem{example}{Example}[section]
\newtheorem{definition}[problem]{Definition}

\newcommand{\BEQA}{\begin{eqnarray}}
\newcommand{\EEQA}{\end{eqnarray}}
\newcommand{\define}{\stackrel{\triangle}{=}}
\bibliographystyle{IEEEtran}

\providecommand{\mbf}{\mathbf}
\providecommand{\pr}[1]{\ensuremath{\Pr\left(#1\right)}}
\providecommand{\qfunc}[1]{\ensuremath{Q\left(#1\right)}}
\providecommand{\sbrak}[1]{\ensuremath{{}\left[#1\right]}}
\providecommand{\lsbrak}[1]{\ensuremath{{}\left[#1\right.}}
\providecommand{\rsbrak}[1]{\ensuremath{{}\left.#1\right]}}
\providecommand{\brak}[1]{\ensuremath{\left(#1\right)}}
\providecommand{\lbrak}[1]{\ensuremath{\left(#1\right.}}
\providecommand{\rbrak}[1]{\ensuremath{\left.#1\right)}}
\providecommand{\cbrak}[1]{\ensuremath{\left\{#1\right\}}}
\providecommand{\lcbrak}[1]{\ensuremath{\left\{#1\right.}}
\providecommand{\rcbrak}[1]{\ensuremath{\left.#1\right\}}}
\theoremstyle{remark}
\newtheorem{rem}{Remark}
\newcommand{\sgn}{\mathop{\mathrm{sgn}}}
\providecommand{\abs}[1]{\left\vert#1\right\vert}
\providecommand{\res}[1]{\Res\displaylimits_{#1}} 
\providecommand{\norm}[1]{\left\lVert#1\right\rVert}
%\providecommand{\norm}[1]{\lVert#1\rVert}
\providecommand{\mtx}[1]{\mathbf{#1}}
\providecommand{\mean}[1]{E\left[ #1 \right]}
\providecommand{\fourier}{\overset{\mathcal{F}}{ \rightleftharpoons}}
%\providecommand{\hilbert}{\overset{\mathcal{H}}{ \rightleftharpoons}}
\providecommand{\system}{\overset{\mathcal{H}}{ \longleftrightarrow}}
	%\newcommand{\solution}[2]{\textbf{Solution:}{#1}}
\newcommand{\solution}{\noindent \textbf{Solution: }}
\newcommand{\cosec}{\,\text{cosec}\,}
\providecommand{\dec}[2]{\ensuremath{\overset{#1}{\underset{#2}{\gtrless}}}}
\newcommand{\myvec}[1]{\ensuremath{\begin{pmatrix}#1\end{pmatrix}}}
\newcommand{\mydet}[1]{\ensuremath{\begin{vmatrix}#1\end{vmatrix}}}

\numberwithin{equation}{subsection}

\makeatletter
\@addtoreset{figure}{problem}
\makeatother
\let\StandardTheFigure\thefigure
\let\vec\mathbf

\renewcommand{\thefigure}{\theproblem}

\def\putbox#1#2#3{\makebox[0in][l]{\makebox[#1][l]{}\raisebox{\baselineskip}[0in][0in]{\raisebox{#2}[0in][0in]{#3}}}}
     \def\rightbox#1{\makebox[0in][r]{#1}}
     \def\centbox#1{\makebox[0in]{#1}}
     \def\topbox#1{\raisebox{-\baselineskip}[0in][0in]{#1}}
     \def\midbox#1{\raisebox{-0.5\baselineskip}[0in][0in]{#1}}
\vspace{3cm}
\title{Assignment 20}
\author{KUSUMA PRIYA\\EE20MTECH11007}

\maketitle
\newpage

\bigskip
\renewcommand{\thefigure}{\theenumi}
\renewcommand{\thetable}{\theenumi}
Download codes from 
%
\begin{lstlisting}
https://github.com/KUSUMAPRIYAPULAVARTY/assignment20
\end{lstlisting}
%
 
\section{QUESTION}
For every $4 \times 4$ real symmetric non-singular matrix $\vec{A}$ there exists a positive integer $p$ such that

    \begin{enumerate}
        \item $p\vec{I}+\vec{A}$ is positive definite
        \item $\vec{A}^p$ is positive definite
        \item $\vec{A}^{-p}$ is positive definite
        \item $\text{exp}(p\vec{A})-\vec{I}$ is positive definite
        \end{enumerate}

%
\section{Theory}
A matrix is real symmetric implies its eigen values are real and eigen vectors are orthogonal,that is its eigen value decomposition is
\begin{align}
 \vec{A}=\vec{P}\vec{D}\vec{P}^T
\end{align}
$\vec{D}$ is the diagonal matrix containing the real eigen values of $\vec{A}$\\
$\vec{P}$ has the corresponding eigen vectors
\begin{align}
    \vec{P}\vec{P}^T=\vec{P}^T\vec{P}=\vec{I}
\end{align}
A real matrix is positive definite if 
\begin{align}
    \vec{x}^T\vec{A}\vec{x}>0\\
    \implies  \vec{x}^T\lambda\vec{x}>0\\
    \implies \lambda \vec{x}^T\vec{x}>0\\
    \implies \lambda>0
\end{align}
In other words, all the eigen values of $A$ are positive
\section{Solution}
Let $\vec{A}$ be
\begin{align}
    \vec{A}=\vec{P}\vec{D}\vec{P}^T\\
    \vec{D}=\myvec{\lambda_1&0&0&0\\0&\lambda_2&0&0\\0&0&\lambda_3&0\\0&0&0&\lambda_4}
\end{align}
\subsection{Choice 1}
\begin{align}
    p\vec{I}+\vec{A}=\vec{P}(p\vec{I})\vec{P}^T+\vec{P}\vec{D}\vec{P}^T\\
   = \vec{P}\vec{D}_1\vec{P}^T\\
    \vec{D}_1=\myvec{\lambda_1+p&0&0&0\\0&\lambda_2+p&0&0\\0&0&\lambda_3+p&0\\0&0&0&\lambda_4+p}
\end{align}
Some of the eigen values of $\vec{A}$ may be negative.\\
All the eigen values in $\vec{D}_1$ are positive only if 
\begin{align}
    p>|\lambda_i|\text{  } \forall i \in [1,4]
\end{align}
\subsection{Choice 2}
\begin{align}
    \vec{A}^2=\vec{A}\vec{A}\\
    =(\vec{P}\vec{D}\vec{P}^T)(\vec{P}\vec{D}\vec{P}^T)\\
    =\vec{P}\vec{D}^2\vec{P}^T\\
    \text{Similarly, }\vec{A}^p=\vec{P}\vec{D}^p\vec{P}^T\\
    \vec{D}^p=\myvec{\lambda_1^p&0&0&0\\0&\lambda_2^p&0&0\\0&0&\lambda_3^p&0\\0&0&0&\lambda_4^p}
\end{align}
Some of the eigen values of $\vec{A}$ may be negative and $\vec{A}^p$ is positive definite only if $p$ is even.
\subsection{Choice 3}
\begin{align}
    \vec{A}^{-p}=\vec{P}\vec{D}^{-p}\vec{P}^T\\
    \vec{D}^{-p}=\myvec{\lambda_1^{-p}&0&0&0\\0&\lambda_2^{-p}&0&0\\0&0&\lambda_3^{-p}&0\\0&0&0&\lambda_4^{-p}}
\end{align}
Some of the eigen values of $\vec{A}$ may be negative and $\vec{A}^{-p}$ is positive definite only if $p$ is even.
\subsection{Choice 4}
\begin{align}
    \text{exp}(p\vec{A})=\sum_{k=0}^\infty \frac{(p\vec{A})^k}{k!}\\
     \implies \text{exp}(p\vec{A})=\vec{P} \text{exp}(p\vec{D})\vec{P}^T\\
     \implies  \text{exp}(p\vec{A})-\vec{I}=\\\vec{P} \text{exp}(p\vec{D})\vec{P}^T-\vec{P}\vec{I}\vec{P}^T\\
     =\vec{P}( \text{exp}(p\vec{D})-\vec{I}) \vec{P}^T\\
     \text{exp}(p\vec{D})-\vec{I}=\\\myvec{e^{\lambda_1}-1&0&0&0\\0&e^{\lambda_2}-1&0&0\\0&0&e^{\lambda_3}-1&0\\0&0&0&e^{\lambda_4}-1}
\end{align}
$\vec{A}$ is non-singular
\begin{align}
   \implies \forall i \in [1,4], \lambda_i\neq0\\
   e^{\lambda_i}<1
\end{align}
So, exp$(p\vec{A})-\vec{I}$ is not positive definite.
\subsection{Answer}
The choices would be option 1,2,3
\end{document}



